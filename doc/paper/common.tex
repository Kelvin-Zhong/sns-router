%HU, Pili
%Create: 20120330
%Modify: 20120330
%The unified entry to include in my tutorial series

%HU, Pili
%Create: 20110910
%Modify: 20120330
%purpose of this file is to gather commonly used
%mathematical abbreviations, to speed up writing
%notes

%\usepackage[utf8x]{inputenc}
%\usepackage{ucs}
%\usepackage{amsmath}
%\usepackage{amsfonts}
%\usepackage{amssymb}
%\usepackage{amsthm}
\usepackage{url}
\usepackage{graphicx}
\usepackage{fancyvrb}

%\usepackage{fancyhdr}
%\pagestyle{fancy}
%\fancyhead{}

%=====Calculus======
%the following commands are not originated by me
%I pick them from http://www-solar.mcs.st-and.ac.uk/~clare/Latex/
%the following line controls the style of patial derivative
%1), use \dfrac, height is larger, looks good. 
%2), use \frac, also work, but space looks limited. 
\newcommand{\myfrac}[2]{\dfrac{#1}{#2}}
\newcommand{\diff}[2]{\myfrac{{\rm d}#1}{{\rm d}#2}}
\newcommand{\ndiff}[3]{\myfrac{{\rm d}^{#3}#1}{{\rm d}#2^{#3}}}
\newcommand{\pdiff}[2]{\myfrac{\partial #1}{\partial #2}}
\newcommand{\npdiff}[3]{\myfrac{\partial^{#3} #1}{\partial #2^{#3}}}
\newcommand{\e}[1]{\ensuremath{{\rm e}^{#1}}}
\newcommand{\ldiff}[2]{\ensuremath{{\rm d}#1/{\rm d}#2}}
\newcommand{\lpdiff}[2]{\ensuremath{\partial#1/\partial#2}}
\newcommand{\lnpdiff}[3]{\ensuremath{\partial^{#3}#1/\partial#2^{#3}}}
\newcommand{\dif}[1]{\mathrm{d}#1}

%20120330
%The reason I don't copy the original file as a whole
%is that it contains too many individually preferred 
%definitions. 
%
%I start with those basic symbols and adapt them in use. 

%=====Matrix======
\newcommand{\tr}[1]{\mathrm{Tr}\left[#1\right]}
\newcommand{\tran}[1]{#1^\mathrm{T}}
\newcommand{\her}[1]{#1^\mathrm{*}}
%The following shorthand of matrix may be convenient. 
%However, it is so short that I'm worried it may 
%collide with something else. I don't use at present.
%\newcommand{\m}[1]{\mathbf{#1}}
\newcommand{\adj}[0]{\mathrm{adj}}
\newcommand{\inv}[1]{#1^{-1}}


%=====Theorem definitions=====
%\newcounter{mytheoremorder}
%\newtheorem{mydef}{Definition}
%\newtheorem{myaxm}{Axiom}
%\newtheorem{mylm}{Lemma}
%\newtheorem{mythm}[mytheoremorder]{Theorem}
%\newtheorem{myprop}[mytheoremorder]{Proposition}
%\newtheorem{myex}{Example}

%=====Optimization====
\DeclareMathOperator*{\argmax}{arg\,max}
\DeclareMathOperator*{\argmin}{arg\,min}
\DeclareMathOperator*{\minimize}{minimize}
\DeclareMathOperator*{\maximize}{maximize}
%\newcommand{\maximize}[0]{\mathrm{Maximize~}}
%\newcommand{\minimize}[0]{\mathrm{Minimize~}}

%=====Probability====
\newcommand{\E}[0]{\mathbb{E}}
\newcommand{\var}[0]{\mathrm{Var}}
\newcommand{\cov}[0]{\mathrm{Cov}}

%=====Quick and Unified Reference====
%20120505
%Usage: \eq{\ref{xxx}}
%The reason I keep "\ref" away from definition, 
%and let user type it every time is that: 
%currently I'm working on Texmaker, and it can 
%trigger a selection panel when the sequence 
%"\ref" is found. Maybe further configuration of 
%texmake can make it do the same thing when the 
%following self-defined sequence is detected.
%This is left for future work. 
\newcommand{\req}[1]{\textbf{Eq~{#1}}}
\newcommand{\rfig}[1]{\textbf{Fig~{#1}}}
\newcommand{\rtbl}[1]{\textbf{Tbl~{#1}}}
\newcommand{\rpg}[1]{\textbf{P~{#1}}}
\newcommand{\rsec}[1]{\textbf{Section~{#1}}}
\newcommand{\ralg}[1]{\textbf{Alg~{#1}}}

